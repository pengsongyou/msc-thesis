\chapter{Implementation details}\label{appendix:implement}

\begin{enumerate}
    \item Detail about how to build Laplacian efficiently inside the mask
    \begin{itemize}
        \item Define a offset vector, which define the offset index of the neighboring pixels centering in (x,y).
        \item For each certain direction neighbour, we have all their index $iN$
        \item Find the index inside $iN$ but not in $imask$, which is right outside the boundary of mask (change iN)
        \item Also move those pixel inside the mask with the neighouring outside (change imask)
        \item directly calculate the weight between (x,y) and its neighbouring, this can be done for all the modified imask
        \item after performing to all 4 neighors, we can build a sparse matrix directly.
    \end{itemize}
    
    
    \item the derivation of $\Psi z = 0$ in RGB ratio model part
    \item List of mathmetical symbol like paper "Shading-based Refinement on Volumetric Signed Distance Functions"
\end{enumerate}
