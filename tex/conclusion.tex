\chapter{Conclusion and Future Work} \label{chap:conclusion}

%%%%%%%%%%%%%%%%%%%%%%%%%%%%%%%%%%%%%%%%%%%%
talk about that almost all the state-of-the-art method in single depth image estimation is not really theoretically correct. Their results looks good but actually not really correct because of the albedo estimation is not satisfying with all those regularizers.
Recently some researchers have proposed a general framework to solve deblurring and demosaiking problems without knowing what the regularizer itself is.
Instead, they separate the classic $\lVert Ax - b\rVert^2 + R(x)$ using methods like Primal-Dual, ADMM or forward backward. To solve the proximal operator of the $R(x)$ in these optimization method, they just solve it with a BM3D denoiser\cite{heide2014flexisp} or a deep denoising neural network\cite{meinhardt2017learning}.

Therefore, it would be very interesting if we can use such a method to calculate the albedo. 


Similar to the p36 book from Forsythe and Ponce, we can times a matrix to deal with the shadow problem in images. 