%%%%%%%%%%%%%%%%%%%%%%%%%%%%%%%%%%%%%%%%%%%%%%%%%%%%%%%%%%%%%%%%%%%%%%%%%%%
% This is a sample header for a sample dissertation. Fill in the name,
% and the other information. LaTeX will work out the table of
% content, the list of figures and of tables for you.
%%%%%%%%%%%%%%%%%%%%%%%%%%%%%%%%%%%%%%%%%%%%%%%%%%%%%%%%%%%%%%%%%%%%%%%%%%%

\newpage
\thispagestyle{empty}

% ******* Title page *******
% **************************

\vspace*{-2cm}
\begin{center}
{\LARGE\bf  High Quality Shape from a RGB-D Camera using Photometric Stereo\\} \vspace{2cm} {\Large \textbf{Songyou Peng}}\\
\vspace{1.5cm}
{
\large 
Supervised by:\\ \vspace{0.4cm}
 Dr. Yvain Qu\'eau \hspace{0.8cm} Prof. Daniel Cremers
}
\\\vspace{0.8cm}
{
\includegraphics[height=0.15\textheight]{figures/tum_logo.jpg}
\\\vspace{0.8cm}
\large
Computer Vision Group \\
\vspace{0.4cm}
Department of Computer Science\\
\vspace{0.4cm}
Technical University of Munich}

\end{center}

\vspace{2cm}
\begin{center}
{
\includegraphics[height=0.1\textheight]{figures/vibot_logo_transparent.png}
\\
\large A Thesis Submitted for the Degree of \\MSc Erasmus Mundus
in Vision and Robotics (VIBOT) \\\vspace{0.3cm} $\cdot$ 2017
$\cdot$}
\end{center}
\singlespacing


%ABSTRACT
\begin{abstract}
Low-cost RGB-D cameras are playing an increasingly important role in many computer vision tasks, however, their captured depth maps not only do not contain fine details of the objects but also have noisy or missing information.
This dissertation proposes two novel methods which can refine the rough depth images based on the theory of photometric stereo.

The first method called RGB ratio model can resolve the nonlinearity problem in most previous methods and promise a closed-form solution.
Modern depth refinement approaches usually could not separate the shape from the complicated albedo, which leads to visible artefacts on the refined depth.
We propose another robust multi-light method which offers the advantage of recovering the real shape from the imperfect depth without any regularization, and it outperforms the state-of-the-art methods.
%which has the capability of enhancing the imperfect depth without any regularization.
% and outperforms state-of-the-art methods.
Moreover, we combine our approach with image super-resolution such that the high-quality and high-resolution depth can be acquired. 
Quantitative and qualitative experiments have demonstrated the robustness and effectiveness of the suggested methods.

\vspace*{5cm}



\begin{center}
\begin{quote}
\it So you have to trust that the dots will somehow connect in your future. You have to trust in something --- your gut, destiny, life, karma, whatever. This approach has never let me down, and it has made all the difference in my life.
\end{quote}
\end{center}
\hfill{\small  --- Steve Jobs}

\end{abstract}

\doublespacing

%\pagestyle{empty}
\pagenumbering{roman}
\setcounter{page}{1} \pagestyle{plain}


\tableofcontents

\listoffigures
\listoftables

\chapter*{Acknowledgments}
\addcontentsline{toc}{chapter}
         {\protect\numberline{Acknowledgments\hspace{-96pt}}}

First of all, I would love to thank my supervisor Prof. Dr. Daniel Cremers for offering me this master thesis opportunity in the Computer Vision Group. 
It is a great honour to be a member of your group. 
This thesis would not have been possible without the continuous support from my advisor Dr. Yvain Qu\'{e}au, who always believed me and gave my so many constructive suggestions. 
Thank you for your patience, effort and trust.

I really appreciate the enormous help offered by one my best friends Tianming Qiu.
My stay in Munich would have been much harder without you.
My gratitude also goes to Nan Yang, Hidenobu Matsuki, Yuesong Shen and all other masters students sharing the office 02.09.038 for your warm company and motivation. 
Besides, I am really grateful to Cancen Jiang and David Kong for your careful proofreading and valuable advice.

I am much obliged to Prof. Peter Sturm, who supervised my summer internship in INRIA during the summer of 2016 and recommended me here for the thesis.
Your rigorous research attitude and extraordinary personality have set a great example to me in all respects.

I am particularly thankful to my VIBOT colleagues who made these wonderful two years one of the most special and satisfactory memories in my life.
All the fun times, all the places we have visited, and the projects, assignments and exams we have gone through together will remain dear to me.
%You will reconize yourselves: Kaisar, Jose, \`{E}ric, Oleksii, Duncan, Yogesh, Gourab, Raabid, Kibrom, Rami, Paola, Daudt. 

Most important of all, my dearest thanks goes to my beloved family.
You have been unconditionally supporting of my decisions and have given me everything. I love you so much.

Songyou, well done, I am proud of you.

\pagestyle{fancy}
